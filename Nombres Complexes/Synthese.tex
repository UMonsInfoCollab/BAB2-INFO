\documentclass[]{article}
\usepackage[french]{babel}
\usepackage[utf8]{inputenc}
\usepackage{amsmath, amsfonts, amssymb, amsthm}

\newcommand{\cis}{\text{cis }}
\newcommand{\cisns}{\text{cis}}

\title{Nombres complexes}
\date{19-10-2021}
\author{Arnaud Moreau}

\begin{document}
\maketitle
\newpage

\noindent
\section{Introduction}
\noindent
Résoudre $ax^2+bx+c=0$, mais dans quoi ?
\begin{enumerate}
\item{Dans $\mathbb{R}$}
\item{Dans $\mathbb{C}$}
\item{...}
\end{enumerate}
$x^2 = 2$ possède des solutions dans $\mathbb{R}$ mais pas dans $\mathbb{N}$.
\\ De même, $x^2 = -1$ possède des solutions dans $\mathbb{C}$ mais pas dans $\mathbb{R}$.
\\ L'ensemble des nombres complexes est $\mathbb{C} = \{z \ | \ z = a+bi, \ a, b \in \mathbb{R}\}$


\section{Résolution des équations du second degré}
\noindent
Dans $\mathbb{R}$ :
\begin{enumerate}
\item{$x_1 = \frac{-b-\sqrt{\Delta}}{2a}$}
\item{$x_2 = \frac{-b+\sqrt{\Delta}}{2a}$}
\end{enumerate}
Dans $\mathbb{C}$ :
\begin{enumerate}
\item{$x_1 = \frac{-b-i\sqrt{|\Delta|}}{2a}$}
\item{$x_2 = \frac{-b+i\sqrt{|\Delta|}}{2a}$}
\end{enumerate}

\section{Notation d'un nombre complexe}
\noindent
\[
z = a+bi \longrightarrow (a, b) \in \mathbb{R}^2
\]
Le nombre complexe peut être représenté dans le plan de Gauss, où l'axe $OX$ représente la partie réelle ($a$) et l'axe $OY$ la partie imaginaire ($b$) du nombre.
\\ $Re(z)$ représente la partie réelle de z.
\\ $Im(z)$ représente la partie imaginaire de z.
\[
|z| \ = \ |a+bi| \ = \ \sqrt{Re(z)^2+Im(z)^2}
\]
Le conjugué de a+bi, noté $\overline{a+bi}$, vaut $a-bi$ et est équivalent à une symétrie d'axe $OY$ sur le plan de Gauss et le cercle trigonométrique.


\section{Opérations sur les complexes}

\subsection{Addition}
\noindent
Soient $z_1$, $z_2 \in \mathbb{C}$.
\begin{align*}
z_1 + z_2 \ &= \ (a_1+b_1i)+(a_2+b_2i)
\\          &= \ (a_1+a_2)+(b_1+b_2)i
\end{align*}

\subsection{Multiplications et puissances}
\noindent
Soit $z \in \mathbb{C}$.
\\
\[
z^1 = z
\]
\[
\forall \ n \in \mathbb{N}, \ z^n = z*z*z*...*z
\]
\[
\forall \ n, \ m \in \mathbb{N}, \ z^n z^m = z^{n+m} \text{ and } z^{n^m} = z^{nm}
\]
\[
z^0 = 1 \Leftarrow z^0 * z^m = z^{0+m} = z^m
\]

\subsection{Inverse}
\noindent
Soit $z \in \mathbb{C}, \ z = a+bi$.
\\
\[
z^{-1} = \frac{a-bi}{a^2+b^2}
\]


\section{Argument}
\noindent
On peut représenter un nombre complexe sur un cercle trigonométrique en utilisant l'axe $OX$ pour représenter $Re(z)$ et l'axe $OY$ pour représenter $Im(z)$.
\\ L'angle obtenu (noté $\Theta$) est appelé l'argument du nombre complexe.
\\
\\ Soit $z \in \mathbb{C}, \ z = a+bi$.
\\
\[
\Theta = arg(z)
\]
\[
\cos{\Theta} = \frac{a}{|z|} \text{ et } \sin{\Theta} = \frac{b}{|z|}
\]
Propriété :
\\
\[
arg(z) = arg\left(\frac{z}{|z|}\right)
\]


\section{Seconde notation d'un nombre complexe}
\noindent
Soit $z \in \mathbb{C}, \ z \neq 0$.
\\
\\ On peut écrire $z = \rho \ \cis (\Theta)$ avec $\rho = |z|, \ \Theta = arg(z)$,
\\ Avec $\cis{\Theta} = \cos{\Theta} + i \sin{\Theta}$
\\
\\ On a donc $\rho \in \mathbb{R}^{>0}$ et $\Theta \in [0; 2 \pi[$.


\section{Opérations avec la nouvelle notation}

\subsection{Addition}
\noindent
Soient $z_1, z_2 \ = \ \rho_1 \ \cis{\Theta_1}, \ \rho_2 \ \cis{\Theta_2}$
\begin{align*}
z_1 + z_2 &= \rho_1\cis\Theta_1+\rho_2\text{cis}\Theta_2
\\ &= \rho_1\cos{\Theta_1}+\rho_2\cos{\Theta_2} + i(\rho_1\sin{\Theta_1}+\rho_2\sin{\Theta_2})
\end{align*}
\begin{proof}
\begin{align*}
z_1 + z_2 &= \rho_1 \ \cis{\Theta_1} + \rho_2 \ \cis{\Theta_2}
\\        &= \rho_1(\cos{\Theta_1}+i\sin{\Theta_1})+\rho_2(\cos{\Theta_2}+i\sin{\Theta_2})
\\        &= \rho_1\cos{\Theta_1}+\rho_2\cos{\Theta_2} + i(\rho_1\sin{\Theta_1}+\rho_2\sin{\Theta_2})
\end{align*}
\end{proof}

\subsection{Multiplications et puissances}
\noindent
Soient $z_1, z_2 \ = \ \rho_1 \ \cis{\Theta_1}, \ \rho_2 \ \cis{\Theta_2}$
\begin{align*}
z_1 z_2 &= (\rho_1\cis\Theta_1)(\rho_2\cis{\Theta_2})
\\ &= \rho_1\rho_2\cisns{(\Theta_1+\Theta_2)}
\end{align*}
\begin{proof}
\begin{align*}
z_1 z_2 &= (\rho_1\cis\Theta_1)(\rho_2\text{cis}\Theta_2)
\\ &= \rho_1(\cos{\Theta_1}+i\sin{\Theta_1})\rho_2(\cos{\Theta_2}+i\sin{\Theta_2})
\\ &= \rho_1\rho_2(\cos{\Theta_1}\cos{\Theta_2}+i^2\sin{\Theta_1}\sin{\Theta_2}+i(\sin{\Theta_1}\cos{\Theta_1}+\cos{\Theta_2}\sin{\Theta_2}))
\\ &= \rho_1\rho_2(\cos{(\Theta_1+\Theta_2)}+i\sin{(\Theta_1+\Theta_2)})
\\ &= \rho_1\rho_2\cisns{(\Theta_1+\Theta_2)}
\end{align*}
\end{proof}
\noindent
Cas particulier : $z_1 = z_2 = z$.
\begin{align*}
z*z &= z^2
\\  &= \rho^n\cisns{(n\Theta)}
\end{align*}
\begin{proof}
\begin{align*}
z*z &= z^2
\\  &= (\rho\cis{\Theta})^2
\\  &= (\rho\cis{\Theta})(\rho\cis{\Theta})
\\  &= \rho\rho(\cis{\Theta}+\cis{\Theta})
\\  &= \rho^2(2\cis{\Theta})
\\  \text{Par récurrence :}
\\  &= \rho^n\cisns{(n\Theta)}
\end{align*}
\end{proof}

\section{L'ensemble $U_n$}
\noindent
\begin{align*}
U_n &= \{z \in \mathbb{C} | z^n = 1\}
\\  &= \{\cis{0}, \cis{\frac{2\pi}{n}}, \ \hdots \ , \cis{\frac{(n-1)2\pi}{n}} \}
\end{align*}
\begin{proof} \
\\
\\ Regardons $x^n = 1$.
\\ Soit $u \in \mathbb{C}$ solution de $x^n = 1$.
\\ $\Rightarrow u^n = 1$
\\ $\Rightarrow |u^n| = |1|$
\\ $\Rightarrow |u|^n = 1$ par propriété de la norme ($\forall \ v \in \mathbb{C}, \forall \ n \in \mathbb{R}, \ |v^n| = |v|^n$)
\\ $\Rightarrow |u| = 1$
\\ $\Rightarrow |u| = \rho = 1$ par définition de $\rho$
\\ $\Rightarrow$ La solution de $x^n = 1$ se trouve sur le cercle trigonométrique
\\ $\Rightarrow z $ est de la forme $\cis{\Theta}$ 
\\
\\ Cherchons $\Theta$.
\\ Nous savons que $(\cis{\Theta})^n = 1$.
\begin{align*}
(\cis{\Theta})^n &= 1
\\ \cisns{(n\Theta)} &= \cis{0}
\\ n\Theta &= k2\pi \text{, } k \in \mathbb{N} \Leftarrow \text{(} n\Theta \text{) est un multiple de } 2\pi \text{.}
\\ \Theta &= \frac{k2\pi}{n}
\\ &\Rightarrow \Theta \in \{ 0, \frac{2\pi}{n}, 2\frac{2\pi}{n}, \ \hdots \ , \frac{(n-1)2\pi}{n}  \}
\\ & \text{Remarque : } \Theta = \frac{n2\pi}{n} \Rightarrow \Theta = 2\pi \text{, or, } \Theta \in [0; 2\pi[ \text{, donc, } \frac{n2\pi}{n} \not\in E \text{.}
\end{align*}
$\Longrightarrow$ Les solutions de l'équation $x^n = 1$ sont celles de la forme $\cis{0}, \cis{\frac{2\pi}{n}}, \ \hdots \ , \cis{\frac{(n-1)2\pi}{n}}$.
\\ $\Longrightarrow$ Les solutions de l'équation $x^n = 1$ appartiennent à l'ensemble $U_n$.
\\ Mathématiquement : $\forall \ z \in \mathbb{C} \ | \ z^n = 1, \ z \in U_n$.
\\ $\Longrightarrow$ C'est la définition de l'ensemble $U_n$.
\end{proof}

\subsection{Propriétés et propositions}
\noindent
Propriété 1. $\forall \ z_0, z_1 \in \mathbb{C} \ | \ z_0$ est solution de $x^n = z, z \in \mathbb{C}$ et $z_1 \in U_n,
\\ z_0z_1$ est solution de $x^n = z$.
\\
\\ Propriété 2. $\forall \ z_0, z_1 \in \mathbb{C} \ | \ z_0, z_1$ sont solutions de $x^n = z, z \in \mathbb{C},
\\ z_1(z_0)^{-1}$ est solution de $x^n = z$.
\\
\\ Proposition 1. Soit  $z_0 \in \mathbb{C} \ | \ z_0$ est solution de $x^n = z, z \in \mathbb{C}$. Alors, les solutions de $x^n = z$ sont $z_0 * u$ où $u \in U_n$.
\end{document}